А) Дано пространство геометрических векторов  $\mathbb{R}^3$, его подпространства  $L_1$  и  $L_2$  и линейный оператор  $\mathcal{A}:\ \ \mathbb{R}^3\rightarrow\mathbb{R}^3$.
Проведите исследование:
\begin{enumerate}
	\item Изобразите на графике подпространства  $L_1$  и  $L_2$.
	\item Методами векторной алгебры составьте формулу для линейного оператора  $\mathcal{A}$.
	\item Составьте его матрицу в базисе $\left\{\vec{i},\vec{j},\vec{k}\right\}$ пространства $\mathbb{R}^3$.
	\item Решите задачу о диагонализации полученной матрицы методом спектрального анализа.
	\item На построенном ранее графике изобразите базис, в котором матрица линейного оператора $\mathcal{A}$ имеет диагональный вид. Объясните его смысл.
\end{enumerate}

$\mathcal{A}$ – оператор отражения пространства  $\mathbb{R}^3$ в $L_1$ параллельно $L_2,$ где $L_1$ задано уравнением $x=0$, $L_2$ – уравнением  $2x=y=-z$.

\vspace{10mm}

\textbf{Решение.}

\textit{It is empty but you can fill it!}

\textit{Ответ}: \textit{It is empty but you can fill it!}
