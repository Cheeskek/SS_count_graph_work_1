Б) Дано пространство $R$ функций, непрерывных на отрезке $[-\pi; \pi]$ со скалярным произведением $(f, g) = \int^\pi_{-\pi} f(t)g(t)dt$ и длиной вектора $\|f\| = sqrt{(f, f)}$.

Тригонометрические многочлены $P_n(t) = \frac{a_0}{2} + a_1 cost + b_1 sin t + \dots + a_n cos nt + b_n sin nt$, где $a_k, b_k$ - вещественные коэффициенты,
образуют подпространство $P$ пространства $R$.

Требуется найти многочлен $P_n(t)$ в пространстве $R$, минимально отличающийся от функции $f(t)$ - вектора пространства $R$.

Указание.
Требуется решить задачу о перпендикуляре: расстояние от $f\left(t\right)$ до $P_n\left(t\right)$ будет наименьшим,
если это длина перпендикуляра $h=f\left(t\right)-P_n\left(t\right)$, опущенного из точки $f\left(t\right)$ на подпространство $P$.
В этом случае, $P_n\left(t\right)$ будет ортогональной проекцией вектора $f\left(t\right)$ на $P$.
Таким образом, требуется найти координаты вектора $P_n\left(t\right)$ (коэффициенты многочлена) в заданном базисе $P$.
Если выбран ортонормированный базис, то эти координаты суть проекции вектора $f\left(t\right)$ на векторы данного базиса.

Проведите исследование:
\begin{enumerate}
	\item Проверьте, что система функций $\left\{1,\cos{t},\sin{t},\ldots\cos{n}t,\sin{n}t\right\}$ является ортогональным базисом подпространства $P$. Нормируйте систему.
	\item Найдите проекции вектора $f\left(t\right)$ (см. варианты) на векторы полученного ортонормированного базиса.
(На вектор $\left\{1\right\}$ найдите проекцию отдельно, а проекции на векторы вида $\left\{\cos{n}t\right\}$ и $\left\{\sin{n}t\right\}$ запишите формулами в зависимости от $n$. Воспользуйтесь свойствами интегралов от четных и нечетных функций на симметрично промежутке.)
	\item Запишите минимально отстоящий многочлен $P_n\left(t\right)$ с найденными коэффициентами (тригонометрический многочлен Фурье для данной функции).
	\item Изобразите (например, в Desmos) графики функции $f\left(t\right)$ и многочлена Фурье различных порядков $n$ (можно положить $n=5;10;15$).
	\item Сделайте вывод о поведении многочлена при росте его порядка.
\end{enumerate}

\[f\left(t\right)=-3t\]

\vspace{10mm}

\textbf{Решение.}

\textit{It is empty but you can fill it!}

\textit{Ответ}: \textit{It is empty but you can fill it!}
