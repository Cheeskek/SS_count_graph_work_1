\section{Задание 3. Линейный оператор и спектральный анализ.}

\textbf{Условие.}

А) Дано пространство геометрических векторов  $\mathbb{R}^3$, его подпространства  $L_1$  и  $L_2$  и линейный оператор  $\mathcal{A}:\ \ \mathbb{R}^3\rightarrow\mathbb{R}^3$.
Проведите исследование:
\begin{enumerate}
	\item Изобразите на графике подпространства  $L_1$  и  $L_2$.
	\item Методами векторной алгебры составьте формулу для линейного оператора  $\mathcal{A}$.
	\item Составьте его матрицу в базисе $\left\{\vec{i},\vec{j},\vec{k}\right\}$ пространства $\mathbb{R}^3$.
	\item Решите задачу о диагонализации полученной матрицы методом спектрального анализа.
	\item На построенном ранее графике изобразите базис, в котором матрица линейного оператора $\mathcal{A}$ имеет диагональный вид. Объясните его смысл.
\end{enumerate}

$\mathcal{A}$ – оператор отражения пространства  $\mathbb{R}^3$ в $L_1$ параллельно $L_2,$ где $L_1$ задано уравнением $x=0$, $L_2$ – уравнением  $2x=y=-z$.

\vspace{10mm}

Б) Дано множество функций  L  и отображение  $\mathcal{A}:\ \ L\rightarrow L$.
Проведите исследование:
\begin{enumerate}
	\item Проверьте, что  L  является линейным пространством над полем $\mathbb{R}$.
	\item Выберите в нём базис.
	\item Убедитесь, что отображение $\mathcal{A}$ является линейным (оператором).
	\item Решите задачу о диагонализации матрицы линейного оператора $\mathcal{A}$ в выбранном базисе методом спектрального анализа:
	\begin{itemize}
        \item в случае, если $\mathcal{A}$ имеет скалярный тип, для диагонализации используйте собственный базис.
        \item в случае, если $\mathcal{A}$ имеет общий тип, для диагонализации используйте жорданов базис (приведите матрицу в жорданову форму).
    \end{itemize}
\end{enumerate}

$L$ – множество многочленов $p\left(x\right)$ степени не выше 2,

$\mathcal{A}\left(p\left(x\right)\right)=\int_{-1}^{1}{K\left(x;y\right)\ p\left(y\right)\ dy}$, где $K\left(x;y\right)=y^2+2x\left(y-1\right)+\left(1-3y^2\right)x^2$.

\vspace{10mm}
\textbf{Решение.}

\textit{It is empty but you can fill it!}

\clearpage
